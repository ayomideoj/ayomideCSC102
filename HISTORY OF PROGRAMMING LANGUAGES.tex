\documentclass{article}

\title{HISTORICAL PERSPECTIVE OF SOME CERTAIN PROGRAMMING LANGUAGES}
\date{3-11-2021}
\author{Ojikutu Ayomide}

\begin{document}
	\maketitle
	
	
		\pagenumbering{gobble}
	\pagenumbering{arabic}
	The programming languages we will be talking about today include:
	\begin{itemize}
		\item python
		\item java
		\item C++
		\item COBOL
		\item BASIC
	\end{itemize}


	\maketitle 
	\section{PYTHON}
Python came into existence in the late 1980s. It was created by Guido Van Rossum. The implementation of python began December 1989. Van Rossum was the head developer of python until 12 July 2018, when he announced his retirement from being the lead developer.

Over the years there have been different versions of python such as:

\begin{itemize}
	\item Python 2.0
	\item Python 3.0
\end{itemize}

\subsection{PYTHONS BENEVOLENT DICTATOR FOR LIFE}
	\begin{itemize}
	\item The python community called Guido Van Rossum pythons benevolent dictator for life because he created the programming language
	
	\item He was born on January 31 in 1956. He was raised in the Netherlands (north west Europe) and got a masters degree in computer science and mathematics at the University of Amsterdam in 1982.
	
	\item While working at the CWI (Centrum Wiskunde and Informatica), Guido helped develop the ABC programming language and came up with the idea for python. He has had various accomplishments and has worked for various institutes for research. 
		\end{itemize}   
	
\subsection{APPLICATION OF PYTHON}
	Python has various application such as:
\begin{itemize}
	\item Web development
	\item Game development
	\item Desktop GUI
	\item Business applications
	\item Audio and video applications
\end{itemize}

It could be possible to develop an operating system that can be centred around python. Due to the fact that it is a programming language there is nothing stopping it from becoming an OS.         


\section{JAVA}

	James Gosling, Mike Sheridan and Patrick Naughton began the development of the java programming language in June 1991. It was then released in 1995 as a core component of S un Microsystems Java platform. Over the years java has been given various names. It was first called oak then green and finally it was named java.

Java was originally designed for interactive televisions, but it was too advanced for the digital television industry in that era i.e it had too much potential and capabilities.

\subsection{FOUNDER}
	Java was developed by James Gosling, Mike Sheridan, and Patrick Naughton but the main creator was James Gosling.
Gosling is was born May 19, 1955. He is often referred to as DR. Java. He was elected a member of the National Academy of engineering in 2004 because of his achievements like the development of java and his contributions to the development of the windows system.
Gosling worked at Sun Microsystem from 1986 to 2010. At Sun he invented an early Unix windowing system called NeWS, which became a lesser-used alternative to the still used X Window, because Sun did not give it an open source license.

	Due to java’s diversity , it has various applications . This can range from:

\begin{itemize}
	\item Desktop GUI Application
	\item Mobile Applications
	\item Web based application
	\item Enterprise applications
	\item Another important application of java is writing network programmes. In fact it is easier to write network programmes on java than in most programming languages.
		\end{itemize}
	
	\subsection{IDE FOR JAVA}
	\begin{itemize}
		\item Eclipse
		\item JDeveloper
		\item Blue J
		\item Net Beans
		
	\end{itemize}
	
	Very few programming languages are related to java. This is due to its uniqueness. Those that are related to it include python, scala and lisp.
	
	\section{C++}
		In 1979, a Danish scientist named  Bjarne Stroustrup began work on C with classes which was the predecessor of C++. In the year of 1982 Bjarne became making a successor to C with classes which gave rise to C++. After years of development , the first edition of C++ was released in 1985.
		
		\subsection{DEVELOPER}
		Bjarne Stroustrup was born on 30 December 1950. He is most notable for the creation and development of C++. He attended Aarhus University 1969–1975 and graduated with a master's degree in mathematics and computer science. He learned the fundamentals of object-oriented programming from its inventor, Kristen Nygaard, who frequently visited Aarhus.
		Over the years Bjarne received several awards for his work such as The Charles Stark Draper Prize, The Computer Pioneer Award and The Faraday Medal.
		
		\subsection{APPLICATION OF C++}
		 \begin{itemize}
			\item Games
			\item GUI based applications
			\item Database software
			\item Operating System
			\item Banking applications
		\end{itemize}
		
		\section{COBOL}
		
	Common Business Oriented Language is a compiled programming language designed mainly for business.

HISTORY
In the late 1950s, their were rising costs for programming. Later on , it was then suggested that if a common business oriented language was used , the cost for programming would be cheaper and faster. 

\subsection{DEVELOPER}
Vassar’s Grace Murray Hopper invented the COBOL programming language  

\subsection{APPLICATION}
The main and only application of COBOL is for business purposes such as online banking and transcation purposes.
A common IDE for COBOL is OpenCobolIDE.

\section{BASIC}
	BASIC (Beginners' All-purpose Symbolic Instruction Code) is a family of general-purpose, high-level programming languages whose design philosophy emphasizes ease of use. 
It was invented by John G. Kemeny and Thomas E. Kurtz of Dartmouth on May 1 ,1964.

\subsection{JOHN AND THOMAS}
John was born on May 31 , 1926 and died on December 26 , 1992. He was a Hungarian born American computer scientist and mathematician.
Thomas Eugene Kurtz (born February 22, 1928) is a retired Dartmouth professor of mathematics and computer scientist, who along with his colleague John G. Kemeny set in motion the then revolutionary concept of making computers as freely available to college students as library books were, by implementing the concept of time-sharing at Dartmouth College. In his mission to allow non-expert users to interact with the computer, he co-developed the BASIC programming language (Beginners All-purpose Symbolic Instruction Code) and the Dartmouth Time Sharing System during 1963 to 1964.

\subsection{APPLICATION}
\begin{itemize}
	\item It is used to solve arithemetic operations
	\item It is used for a variety of business applications
	\item It is used to teach beginner programmers in school 
\end{itemize}

\subsection{IDE FOR BASIC}
\begin{itemize}
	\item Xbasic
	\item Lazarus
	\item MonoDevelop
\end{itemize}


\end{document}

